\documentclass{article}

\title{Using Generative Adversarial Networks to Predict and Generate Superconductive Materials}
\author{Rajeev Atla}
\date{}

\begin{document}

\maketitle

\begin{abstract}
Achieving a room-temperature superconductor is considered the hallmark of condensed matter physics.
However, researchers have faced many challenges in their search for such a material.
Searching through the entire phase diagrams of each of these materials would likely take decades without a way to accelerate the process.
It would be of great assistance to researchers in the field to be able to differentiate between non-superconductors and superconductors through a data-based method.
Recent work has used easily observable characteristics to predict critical temperature.
However, no clear association has been found between the easily observable characteristics of a material and its superconductive phase diagram, so a supervised learning method isn't realizable.
Instead, unsupervised learning must be used.
Herein, we report a solution to this problem: a generative adversarial network (GAN) that allows researchers to evaluate materials to further investigate for possible supercondctivity.
Such a method would also be able to generate possible superconductive materials to be tested. 
\end{abstract}




\end{document}
